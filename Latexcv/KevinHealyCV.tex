% 15/05/2014 
%LaTex version of my CV

\documentclass[10pt,a4paper]{article}
\usepackage[left=0.7in, top=0.6in, right=0.7in, bottom=0.6in]{geometry}
\usepackage{enumerate} % put in numbers or bullet points
\usepackage{setspace}	% line spacing					
\usepackage{eurosym}
%\usepackage{fullpage}
\usepackage{fancyhdr}
\pagestyle{fancy} % page numbers and headers and footers


\fancyhead[RO,RE]{Kevin Healy CV}
\fancyfoot[C]{\thepage~of 3}

\renewcommand{\headrulewidth}{0.1pt}
\renewcommand{\footrulewidth}{0pt}



%\usepackage[osf]{mathpazo} % palatino font package
\usepackage{fontspec} %this must be in XeLaTeX
%\setmainfont{Verdana}%
%\setmainfont{Calibri}%
%\setsansfont{Tahoma}%
\setmainfont{Helvetica}%
%\setmainfont{Garamond}%
\usepackage{hyperref} % allows the inclusion of hyperlinks




\usepackage{titlesec} % Used to customize the \section command
\titleformat{\section}{\Large\raggedright}{}{0em}{} [\titlerule] % Text formatting of sections
\titlespacing{\section}{0pt}{3pt}{10pt} % Spacing around sections (left, above, below)

\begin{document}

\par{\centering{\Huge Kevin {Healy}}\smallskip\par}

\large\centering{{Zoology PhD Candidate at Trinity College Dublin}}\\

\par{\normalsize\centering{{Theoretical evolutionary ecologist with diverse quantitative skills focusing on using comparative analysis and energetic modelling approaches to understand trophic interactions at ecological and evolutionary scales.}}\bigskip\par}


\begin{minipage}[t]{0.5\textwidth}
\raggedright

Email:  healyke@tcd.ie\\
Tel:\hspace{0mm}+353 851557282\\
Address: Zoology Department,\\
\hspace{0mm}School of Natural Sciences,\\
\hspace{0mm}Trinity College Dublin,\\ 
\hspace{0mm}Ireland.\\

\end{minipage}
\begin{minipage}[t]{0.45\textwidth}

Website:\href{http://healyke.github.io}{healyke.github.io}\\
Google Scholar:\href{http://scholar.google.com/citations?user=5Kb9u8EAAAAJ}{http://tinyurl.com/q2wsjsb}\\
ORCID:\href{http://orcid.org/0000-0002-3548-6253}{0000-0002-3548-6253}\\
ResearcherID:\href{http://www.researcherid.com/rid/H-6512-2013}{H-6512-2013}\\
Twitter:\href{https://twitter.com/healyke}{@healyke}\\
\end{minipage}

\bigskip

%-------------------------------------
% Publications
%--------------------------------------
\section{Academic Publications}
\begin{flushleft}
\textbf{Healy, K}., Guillerme T., Finlay, S., Kane, A., Kelly, S.B.A., McClean, D., Kelly, D.J., Donohue, I., Jackson, A.L. and Cooper, N., 2014. Ecology and mode-of-life explain lifespan variation in birds and mammals. \textit{Proceedings of the Royal Society B}, \textbf{281}(1784), 20140298.\\ \href{http://rspb.royalsocietypublishing.org/content/281/1784/20140298.full.pdf?keytype=ref&ijkey=gPt28ElSAYBvRhZ}{DOI: 10.1098/rspb.2014.0298}. Journal Impact Factor:5.68
\smallskip
\par{\fontsize{10.5}{10}\selectfont My second full peer-reviewed paper as lead author. I developed and carried out the main analysis and was heavily involved in the initial conception, data collection and writing of the manuscript. This publication has three google scholar citations placing it in the top 1 percentile of the Environment/Ecology category of papers published in 2014.}

\bigskip

\textbf{Healy, K}., McNally, L, Ruxton, G., Cooper, N. and Jackson, A.L. 2013. Metabolic rate and\\
body size linked with perception of temporal information.  \textit{Animal Behaviour}. \textbf{86}, 685-696. \href{http://dx.doi.org/10.1016/j.anbehav.2013.06.018}{DOI: 10.1016/j.anbehav.2013.06.018}. Journal Impact Factor:3.4
\smallskip
\par{\fontsize{10.5}{10}\selectfont My first full peer-reviewed paper as lead author. I carried out the data collection, statistical analysis and writing of the paper. It was extensively covered in the media, most notably in the New York Times and the BBC. This articel also has the highest ever \href{http://www.altmetric.com/details.php?key=517059da36b98ab7d4941284da32e5f7&citation_id=1705703&embedded=true}{alt-metric score} for this journal.} %link This publication has three google scholar citations placing it in the top 10 percentile of the Environment/Ecology category of papers published in 2013.

\bigskip

\setlength{\parindent}{0mm}Donohue, I., Petchey, O.L., Montoya, J.M., Jackson, A.L., McNally, L., Viana, M.,\textbf{Healy, K}., Lurgi, M., O’Connor, N.E. and Emmerson, M.C. 2013. On the dimensionality of ecological stability. \textit{Ecology Letters}. \textbf{16}, 421-429. \href{http://onlinelibrary.wiley.com/doi/10.1111/ele.12086/abstract}{DOI: 10.1111/ele.12086}.  Journal Impact Factor:17.95\\
\smallskip
\par{\fontsize{10.5}{10}\selectfont This was my first full peer-reviewed paper. I was involved in both the development of the conceptual framework during a three-day workshop and also in the development of the statistical analysis used to produce the multidimensional ellipsoids.This publication has eleven google scholar citations placing it in the top 10 percentile of the Environment/Ecology category of papers published in 2013.}
\end{flushleft}

%----------------------------------------------------------------------------------------
%	EDUCATION
%----------------------------------------------------------------------------------------

\section{Education}

\raggedright	
\textbf{2011-Present:} PhD in Zoology, Trinity College Dublin. Title:“General scaling of predator-prey interactions”. Supervised by Dr. Andrew Jackson and Dr. Andrew Parnell. \smallskip
\par{\fontsize{10.5}{10}\selectfont I investigated how various ecological and physiological traits such as sensory perception, manoeuvrability and lifespan define the ability of individuals to interact with one another. I focused on traits associated with scaling relationships such as mass and metabolic rates and how these traits link towards more complex structures across levels from individuals to ecosystem function.To date I have two peer-reviewed publications in international journals directly arising from my PhD project along with an addition publication on ecosystem stability arising form my involvement in several workshops.\bigskip}

%------------------------------------------

\textbf{2007-2011:} B.A. Mod in Zoology, First class honours and Gold Medal, Trinity College Dublin\\
Thesis: "Fractal structure of intestinal parasite communities in the field mouse".\\
\par{\fontsize{10.5}{10}\selectfont Projected showing that the distribution of intestinal parasite body sizes follows a distribution predicted from the fractal structure of the mouse intestine. (Overall mark of 82\%).\bigskip}

%------------------------------------------
\textbf{2010:} Ureka research position in SoMER (Summer of Molecular Evolution Research) program, National College of Ireland Maynooth.\\
\par{\fontsize{10.5}{10}\selectfont Ten week program under the supervision of Dr. Christen Griffen researching the evolutionary divergence of several morphs of the entomopathogenic nematode species by developing sequencing techniques to investigate the divergence of nematode morphs and their associated symbiotic bacteria strains.\bigskip}


%	Awards and Grants
%----------------------------------------------------------------------------------------

\section{Awards and Grants}

\begin{tabular}{ll}
\textbf{2014:} & Gordon Research Seminar mentoring program position. Funded by Gordon\\
& Research Conferences and the National Science Foundation. (\euro 1,300)\\
\textbf{2014:} & Named top contributer to TCD ecology and evolution discussion group "NERD club".\\
\textbf{2011:} & Ph.D. TCD in Theoretical Ecology. Funded by the HEA through the PRTLI-5 and\\
& co-funded by the ERDF. (\euro 90,000)\\
\textbf{2011:} & Awarded Gold medal by TCD for “exceptional merit at degree examinations”\\
& in final year of B.A Mod. Zoology by coming first in class and achieving an\\
& overall final year mark of 77\%.\\
\textbf{2010:} & Ureka research position in SoMER (Summer of Molecular Evolution Research) \\
& program National College of Ireland Maynooth.  Funded by Science Foundation\\
& Ireland. (\euro 3000)\\ 
\end{tabular}

\bigskip

%-------------------------------------
% Conferences and workshops
%--------------------------------------


\section{Conference presentations}

\raggedright
\begin{tabular}{ll}
\textbf{2014:} & Gordon Research Seminar "Unifying Ecology Across Scales".\\ 
& \textsl{Talk}:"A tail of two extremes".\\
&\textsl{Poster}:"Ecology and mode-of-life explain lifespan variation in birds and mammals".\\
%-------------------------------
\textbf{2014:} & Invited talk for the Irish Longitudinal Study on Aging (TILDA).\\ 
& \textsl{Talk}:"Ecology and mode-of-life explain lifespan variation in birds and mammals".\\
%-------------------------------
\textbf{2014:} & Keynote student talk at BES Macroecology meeting Nottingham.\\ 
& \textsl{Talk}:"Ecology and mode-of-life explain lifespan variation in birds and mammals".\\
%-------------------------------
\textbf{2014:} & University College Dublin Earth Institute Industry and enterprise showcase.\\
&\textsl{Poster}:"Ecology and mode-of-life explain lifespan variation in birds and mammals".\\
%-------------------------------
\textbf{2013:} & ESEB XIV Congress, Lisbon, Portugal: Fifteen minute talk and poster presentation.\\
&\textsl{Talk}:"Metabolic rate and body size linked with perception of temporal information"\\
& \textsl{Poster}:"Ecology and mode-of-life explain lifespan variation in birds and mammals".\\
%-----------------------------
\textbf{2013:} & British Ecological Society Macroecology SIG meeting.\\
&\textsl{Talk}:"Metabolic rate and body size linked with perception of temporal information".\\
%-------------------------------
\textbf{2013:} & University College Dublin Earth Institute Industry and enterprise showcase.\\
&\textsl{Poster}:"Metabolic rate and body size linked with perception of temporal information".\\
%-------------------------------
\textbf{2013:} & Trinity College Dublin Zoology and Botany Postgraduate Symposium.\\
&\textsl{Talk}:"Metabolic rate and body size linked with perception of temporal information".\\
%-------------------------------
\textbf{2012:} & IsoEcol: International Conference on Applications of Stable Isotope Techniques\\
&to Ecological Studies, Brest, France.\\
 &\textsl{Talk}:"Accounting for the process of foraging in source-level variation in isotopic\\
& mixing models".\\
%-------------------------------


\end{tabular}
\bigskip

%-------------------------------------
% Workshops
%--------------------------------------
\section{Workshops}

\begin{tabular}{ll}
%---------------------------------------
\textbf{2014:} & Tansley Workshop: Collaborative meeting to develop metrics to measure ecosystem\\
&multistabilty, Silwood Park, Imperial College London.\\
%---------------------------------------
\textbf{2014:} & Software Carpentry Workshop covering Unix, Git repositories and creating\\
&R packages, University of Nottingham.\\
%---------------------------------------
\textbf{2013:} & Spatial Analysis in R Workshop, Barry Rowlingson, University of Sheffield.\\
%------------------
\textbf{2013:} & Introduction to Morphometrics Workshop, François Gould, Trinity College Dublin.\\
%---------------------
\textbf{2013:} & IUCN Red List of Ecosystems Workshop, Edmund Barrow, Trinity College Dublin.\\
%--------------------
\textbf{2012:} & Introduction to Bayesian analysis using WinBugs, David Lund, University of Cambridge.\\
%--------------------
\textbf{2012:} & Innovation Academy Creative thinking workshop, Trinity College Dublin.\\
%--------------------
\textbf{2012:} & Innovation Academy Film production workshop, Trinity College Dublin.\\
%--------------------
\textbf{2012:} & Introduction to the website management software DreamWeaver, Trinity College Dublin.\\
%--------------------
\textbf{2011:} & Introduction to Stable Isotope Mixing models, Andrew Jackson, Trinity College Dublin.\\
%--------------------
\textbf{2009:} & Mayfly Identification workshop, Mary Kelly Quinn, National Biodiversity Data Centre.\\
%--------------------
&\\
\end{tabular}

%-------------------------------------
% Skills
%--------------------------------------
\bigskip
\section{Skills}

\raggedright\textbf{Quantitative skills}\\

\begin{tabular}{ll}
%---------------------------------------
\textbullet & Modelling and Statistical analysis in R, for example phylogenetic comparative analysis\\
&using both likelihoods (PGLS) and Bayesian based (MCMCglmm) approaches.\\
\textbullet & Bayesian modelling using BUGS software and High performance computing using\\
&UNIX based parallel computing in the Trinity Centre for High performance Clusters.\\
\textbullet & Individual based modelling using Netlogo software.\\

&\\
\end{tabular}
\raggedright\textbf{Field, Communication and Laboratory skills}\\

\begin{tabular}{ll}
%---------------------------------------
\textbullet & Typesetting using LaTex.\\
\textbullet & Version control and data sharing: GitHub and Figshare accounts.\\
\textbullet & Graphics software including Inkscape, GIMP and ImageJ.\\
\textbullet & Experience writting press releases and interacting with both broadcasting media,\\ 
&e.g. interviewed on BBC4 World News and RTE radio, and print media, e.g. articles\\
&covering my reaserch in the Guardian and the New Yorker.\\
\textbullet & Fieldwork experience in small mammal trapping, parasitic helminth identification\\ 
& and archaeological excavation.\\
\textbullet & Molecular techniques including PCR and AFLAP gained during UREKA program.\\
&\\
\end{tabular}

\bigskip



%----------------------------------------------------------------------------------------
%	Academic outreach
%----------------------------------------------------------------------------------------

\section{Academic service and outreach}
\raggedright\textbf{Outreach}\\
\begin{tabular}{ll}
%---------------------------------------
\textbullet& I organised a \href{http://discoverresearchdublin.com/events/zoological-museum/}{Discover Research Night} event in the TCD Zoology Museum aimed at\\ 
&communicating and evolution research to the general public which was estimated to\\
&have an attendance of over 200.\\
\textbullet & I regularly produce research Videos and Images aimed at both the scientific and\\
& the general public \href{http://healyke.github.io/outreach.html}{see website}.\\
\textbullet &I am a regular contributor to the colleges\href{http://www.ecoevoblog.com/}{ EcoEvo blog} with one of my posts\\
& reaching the semifinal stages of the \href{http://www.3quarksdaily.com/3quarksdaily/2014/09/3qd-science-prize-semifinalists-2014.html}{3 quirks daily science blog awards}.\\
\textbullet & Involved in numerous outreach events including BioBlitz (public cataloging of biodiversity)\\
&GameJam (Welcome Trust), Soapbox Science and I'm a scientist get me out of here.
\end{tabular}

\raggedright\textbf{Professional society membership}\\
\begin{tabular}{ll}
%---------------------------------------
\textbullet& European Society for Evolutionary Biology (ESEB) and British Ecological Society (BES)\\ % I need more.

\end{tabular}

\raggedright\textbf{Reviewing}\\
\begin{tabular}{ll}
%---------------------------------------
\textbullet&I regularly act as a reviewer for several top academic journals including the Journal\\
&of Animal Ecology, the Journal of Biogeography and Proceedings of the Royal Society B. \\ 

\end{tabular}



\bigskip

\raggedright\textbf{Teaching Experience}\\
\begin{tabular}{ll}
%---------------------------------------
\textit{Teaching and Toturials}:&I have run several R and statistical help workshops and classes of\\
& research comprehension course for Senior Sophister Zoology class.\\

\end{tabular}


\begin{tabular}{ll}
\textit{Field Course Assistant}:& Field course assistant for week long intensive terrestrial course for\\ 
&  Junior Sophister teaching field skills in small mammal trapping,\\
&insect and bird identification and general field skills\\
\end{tabular}

\begin{tabular}{ll}
\textit{Project subervision}:&\hspace{7.5mm}Co-supervision of Senior Sophister Zoology student thesis project\\
&\hspace{7.5mm}entitled “Fractal structure of intestinal parasite communities”.\\

\end{tabular}

\begin{tabular}{ll}
\textit{Demonstrating}:&\hspace{15mm}Lab demonstrating in Data Handling, Fundamentals in ecology\\ 
&\hspace{15mm}Physiology and Botany (Senior Freshman).\\

\end{tabular}

\bigskip
%-------------------------------------
% References
%--------------------------------------
\section{References}
Contact for References


\bigskip

\end{document}