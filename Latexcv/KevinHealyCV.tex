% 15/05/2014 
%LaTex version of my CV

\documentclass[10pt,a4paper]{article}
\usepackage[left=0.7in, top=0.6in, right=0.7in, bottom=0.6in]{geometry}
\usepackage{enumerate} % put in numbers or bullet points
\usepackage{setspace}	% line spacing					
\usepackage{eurosym}
%\usepackage{fullpage}
\usepackage{fancyhdr}
\pagestyle{fancy} % page numbers and headers and footers


\fancyhead[RO,RE]{Dr Kevin Healy CV}
\fancyfoot[C]{\thepage~of 4}

\renewcommand{\headrulewidth}{0.1pt}
\renewcommand{\footrulewidth}{0pt}

\renewcommand*{\familydefault}{\sfdefault} %% I changed the font
\usepackage{helvet}

%\usepackage[osf]{mathpazo} % palatino font package
%\usepackage{fontspec} %this must be in XeLaTeX
%\setmainfont{Verdana}%
%\setmainfont{Calibri}%
%\setsansfont{Tahoma}%
%\setmainfont{Helvetica}%
%\setmainfont{Garamond}%
\usepackage{hyperref} % allows the inclusion of hyperlinks




\usepackage{titlesec} % Used to customize the \section command
\titleformat{\section}{\Large\raggedright}{}{0em}{} [\titlerule] % Text formatting of sections
\titlespacing{\section}{0pt}{3pt}{10pt} % Spacing around sections (left, above, below)

\begin{document}

\par{\centering{\Huge Dr Kevin {Healy}}\smallskip\par}

\large\centering{{Research Fellow in Zoology at Trinity College Dublin}}\\

\par{\normalsize\centering{{Theoretical evolutionary ecologist with diverse quantitative skills including comparative analysis and metabolic modelling approaches and an interest in life-history and trophic interactions at ecological and evolutionary scales.}}\bigskip\par}


\begin{minipage}[t]{0.5\textwidth}
\raggedright

Email:  healyke@tcd.ie\\
Tel:\hspace{0mm}+353 851557282\\
Address: Zoology Department,\\
\hspace{0mm}School of Natural Sciences,\\
\hspace{0mm}Trinity College Dublin, Ireland.\\ 


\end{minipage}
\begin{minipage}[t]{0.45\textwidth}

Website:\href{http://healyke.github.io}{healyke.github.io}\\
Google Scholar:\href{http://scholar.google.com/citations?user=5Kb9u8EAAAAJ}{http://tinyurl.com/q2wsjsb}\\
ORCID:\href{http://orcid.org/0000-0002-3548-6253}{0000-0002-3548-6253}\\
ResearcherID:\href{http://www.researcherid.com/rid/H-6512-2013}{H-6512-2013}\\
Twitter:\href{https://twitter.com/healyke}{@healyke}\\
\end{minipage}

\bigskip

\section{Education and Academic Career}
\raggedright	
\textbf{2015 - Present} Reserch Fellow in Zoology, Trinity College Dublin.
 \smallskip
\par{\fontsize{10.5}{10}My current position focuses on using quantitative techniques to study variation across species life-history traits using the animal and plant demography datasets COMADRE and COMPADRE. 

Supervised by Prof. Yvonne Buckley (Trinity College Dublin) and Dr Salguero-Gomez (University of Sheffield).\bigskip}

\raggedright	
\textbf{2011 - 2015:} Ph.D. in Zoology, Trinity College Dublin. Title: “Predator-prey allometry across body size and interaction dimensionality”. Supervised by Dr. Andrew Jackson and Dr. Andrew Parnell. \smallskip
\par{\fontsize{10.5}{10} I investigated how various ecological and physiological traits, including visual perception, species lifespan and venom production, define predator-prey interactions. I focused on how these traits scale with size and habitat dimensionality and how this approach can link these processes across ecological and evolutionary scales.\bigskip}

%------------------------------------------

\textbf{2007-2011:} B.A. Mod in Zoology, First class honours and Gold Medal, Trinity College Dublin\\
Thesis: "Fractal structure of intestinal parasite communities in field mice". (Overall mark of 82\%).\bigskip

%------------------------------------------
\textbf{2010:} Ureka research position in SoMER program, National College of Ireland Maynooth.\\
\par{\fontsize{10.5}{10}\selectfont Ten week program under the supervision of Dr. Christen Griffen investigating the evolutionary divergence of entomopathogenic nematodes.\bigskip}


%-------------------------------------
% Publications
%--------------------------------------
\section{Publications}
\begin{flushleft}
\textbf{Healy, K}., Guillerme T., Finlay, S., Kane, A., Kelly, S.B.A., McClean, D., Kelly, D.J., Donohue, I., Jackson, A.L. and Cooper, N., 2014. Ecology and mode-of-life explain lifespan variation in birds and mammals. \textit{Proceedings of the Royal Society B}, \textbf{281}(1784), 20140298. \href{http://rspb.royalsocietypublishing.org/content/281/1784/20140298}{DOI:10.1098/rspb.2014.0298. Link to paper}.
\smallskip
\par{\fontsize{10.5}{10} Lead author. Developed and carried out the main analysis along with data collection and writing of the manuscript. This publication has 14 google scholar citations.}

\bigskip

\textbf{Healy, K}., McNally, L, Ruxton, G., Cooper, N. and Jackson, A.L. 2013. Metabolic rate and\\
body size linked with perception of temporal information.  \textit{Animal Behaviour}. \textbf{86}, 685-696. \href{http://dx.doi.org/10.1016/j.anbehav.2013.06.018}{DOI:10.1016/j.anbehav.2013.06.018. Link to paper}.
\smallskip
\par{\fontsize{10.5}{10} Lead author. Developed and carried out the main analysis along with data collection and writing of the manuscript. Extensively covered in the media with the highest ever \href{http://www.altmetric.com/details.php?key=517059da36b98ab7d4941284da32e5f7&citation_id=1705703&embedded=true}{alt-metric score} for this journal. This publication has 14 google scholar citations.} 

\bigskip

\setlength{\parindent}{0mm}Donohue, I., Petchey, O.L., Montoya, J.M., Jackson, A.L., McNally, L., Viana, M., \textbf{Healy, K}., Lurgi, M., O’Connor, N.E. and Emmerson, M.C. 2013. On the dimensionality of ecological stability. \textit{Ecology Letters}. \textbf{16}, 421-429. \href{http://onlinelibrary.wiley.com/doi/10.1111/ele.12086/abstract} {DOI:10.1111/ele.12086. Link to paper}. 
\smallskip
\par{\fontsize{10.5}{10} I co-developed the conceptual framework and statistical analysis used to produce the multidimensional ellipsoids and contributed to writing the manuscript.This publication has 30 google scholar citations.}
\bigskip

\section{Other Publications}
\textbf{In review}\\
\setlength{\parindent}{0mm}Kane, A., Ruxton, G.D., Jackson, A.L., and \textbf{Healy, K}. 2015. Body size drives importance of scavenging in theropods. \textit{In review AmNat}.
\smallskip
\par{\fontsize{10.5}{10} Senior author. Showed that theropod dinosaurs of intermediate body size are more efficient scavengers than individuals of extreme body sizes by using agent based modelling. I carried out the data collection, analysis and writing of the paper.}

\bigskip

\textbf{Comment response}\\
\setlength{\parindent}{0mm}\textbf{Healy, K}. 2015.  Eusociality but not fossoriality drives longevity in small mammals. \textit{Proceedings of the Royal Society B}, \textbf{282}, 20142917. \href{http://rspb.royalsocietypublishing.org/content/282/1806/20142917} {DOI:10.1098/rspb.2014.2917. Link to paper}. 
\smallskip
\par{\fontsize{10.5}{10} Single author. I carrying out additional analysis in response to a comment on my Healy et al 2014 paper where I show eusociality but not fossoriality is a driver of longevity in mammals.}


\end{flushleft}

%-------------------------------------
% Skills
%--------------------------------------
\bigskip
\section{Skills}

\raggedright\textbf{Quantitative skills}\\

\begin{tabular}{ll}
%---------------------------------------
\textbullet & Modelling and Statistical analysis in R, for example phylogenetic comparative analysis\\
&using both likelihood (PGLS) and Bayesian (MCMCglmm) approaches.\\
\textbullet & Bayesian modelling using JAGS software and High performance computing using\\
&UNIX based parallel computing in the Trinity Centre for High performance Clusters.\\
\textbullet & Individual based modelling using Netlogo software.\\

&\\
\end{tabular}
\raggedright\textbf{Field, Communication and Laboratory skills}\\

\begin{tabular}{ll}
%---------------------------------------
\textbullet & Typesetting using LaTex.\\
\textbullet & Version control and data sharing: GitHub and Figshare accounts.\\
\textbullet & Graphics software; including Inkscape, GIMP and ImageJ.\\
\textbullet & Public speaking; including radio, television and public events.\\ 
\textbullet & Fieldwork experience in small mammal trapping, parasitic helminth identification\\ 
& and archaeological excavation.\\
\textbullet & Molecular techniques including PCR and AFLAP gained during UREKA program.\\
&\\
\end{tabular}

\bigskip




%	Awards and Grants
%----------------------------------------------------------------------------------------

\section{Awards and Grants}

\begin{tabular}{ll}
\textbf{2014:} & Gordon Research Seminar mentoring program position. Funded by Gordon\\
& Research Conferences and the National Science Foundation. (\euro 1,300)\\
\textbf{2014:} & Awarded runner up in both the School of Natural Sciences postgraduate lightning talks\\
& and at the Zoology and Botany postgraduate symposium.\\
\textbf{2014:} & Named top contributer to TCD ecology and evolution discussion group "NERD club".\\
\textbf{2011:} & Ph.D. TCD in Theoretical Ecology. Funded by the HEA through the PRTLI-5 and\\
& co-funded by the ERDF. (\euro 90,000)\\
\textbf{2011:} & Awarded Gold medal by TCD for “exceptional merit at degree examinations”\\
& in final year of B.A Mod. Zoology by coming first in class and achieving an\\
& overall final year mark of 77\%.\\
\textbf{2010:} & Ureka research position in SoMER (Summer of Molecular Evolution Research) \\
& program National College of Ireland Maynooth.  Funded by Science Foundation\\
& Ireland. (\euro 3000)\\ 
\end{tabular}

\bigskip

%-------------------------------------
% Conferences and workshops
%--------------------------------------


\section{Conference and invited presentations}

\raggedright
\begin{tabular}{ll}
\textbf{2015:} & Invited speaker to the Dublin Science Gallery Café Dark Secrets event.\\ 
& \textsl{Talk}: "BIOLUMINESCE: How living organisms produce and emit light"\\

\textbf{2014:} & Gordon Research Seminar "Unifying Ecology Across Scales".\\ 
& \textsl{Talk}: "A tail of two extremes".\\
&\textsl{Poster}: "Ecology and mode-of-life explain lifespan variation in birds and mammals".\\
%-------------------------------
\textbf{2014:} & Keynote student talk at BES Macroecology meeting Nottingham.\\ 
& \textsl{Talk}: "Ecology and mode-of-life explain lifespan variation in birds and mammals".\\
%-------------------------------
\textbf{2014:} & Invited speaker for the Irish Longitudinal Study on Aging (TILDA).\\ 
& \textsl{Talk}: "Ecology and mode-of-life explain lifespan variation in birds and mammals".\\
%-------------------------------
\textbf{2014:} & Invited speaker to the Dublin Science Gallery Café DEAD BEATS event.\\ 
& \textsl{Talk}: "Why so venomous?"\\
%-------------------------------
\textbf{2014:} & University College Dublin Earth Institute Industry and enterprise showcase.\\
&\textsl{Poster}: "Ecology and mode-of-life explain lifespan variation in birds and mammals".\\
%-------------------------------
\textbf{2013:} & ESEB XIV Congress, Lisbon, Portugal\\
&\textsl{Talk}: "Metabolic rate and body size linked with perception of temporal information"\\
& \textsl{Poster}: "Ecology and mode-of-life explain lifespan variation in birds and mammals".\\
%-----------------------------
\textbf{2013:} & British Ecological Society Macroecology SIG meeting.\\
&\textsl{Talk}: "Metabolic rate and body size linked with perception of temporal information".\\
%-------------------------------
\textbf{2013:} & University College Dublin Earth Institute Industry and enterprise showcase.\\
&\textsl{Poster}: "Metabolic rate and body size linked with perception of temporal information".\\
%-------------------------------
\textbf{2013:} & Trinity College Dublin Zoology and Botany Postgraduate Symposium.\\
&\textsl{Talk}: "Metabolic rate and body size linked with perception of temporal information".\\
%-------------------------------
\textbf{2012:} & IsoEcol: International Conference on Applications of Stable Isotope Techniques\\
&to Ecological Studies, Brest, France.\\
 &\textsl{Talk}: "Accounting for the process of foraging in source-level variation in isotopic\\
& mixing models".\\
%-------------------------------


\end{tabular}
\bigskip

%-------------------------------------
% Workshops
%--------------------------------------
\section{Workshops}

\begin{tabular}{ll}
%---------------------------------------
\textbf{2015:} & Methods in Ecology and Evolution Workshop on Open Science\\
& BES, Darwin House London.\\
%---------------------------------------
\textbf{2014:} & Tansley Workshop: Collaborative meeting to develop metrics to measure ecosystem\\
&multistabilty, Silwood Park, Imperial College London.\\
%---------------------------------------
\textbf{2014:} & Software Carpentry Workshop covering Unix, Git repositories and creating\\
&R packages, University of Nottingham.\\
%---------------------------------------
\textbf{2014:} & Integral Projection Models for ecoogical demography, Rob Salguero-Gómez\\
&and Yvonne Buckley, Trinity College Dublin.\\
%---------------------------------------
\textbf{2013:} & Spatial Analysis in R Workshop, Barry Rowlingson, University of Sheffield.\\
%------------------
\textbf{2013:} & Introduction to Morphometrics Workshop, François Gould, Trinity College Dublin.\\
%---------------------
\textbf{2013:} & IUCN Red List of Ecosystems Workshop, Edmund Barrow, Trinity College Dublin.\\
%--------------------
\textbf{2012:} & Introduction to Bayesian analysis using WinBugs, David Lund, University of Cambridge.\\
%--------------------
\textbf{2012:} & Innovation Academy Creative thinking workshop, Trinity College Dublin.\\
%--------------------
\textbf{2012:} & Innovation Academy Film production workshop, Trinity College Dublin.\\
%--------------------
\textbf{2012:} & Introduction to the website management software DreamWeaver, Trinity College Dublin.\\
%--------------------
\textbf{2011:} & Introduction to Stable Isotope Mixing models, Andrew Jackson, Trinity College Dublin.\\
%--------------------
\textbf{2009:} & Mayfly Identification workshop, Mary Kelly Quinn, National Biodiversity Data Centre.\\
%--------------------
&\\
\end{tabular}


\newpage
%----------------------------------------------------------------------------------------
%	Academic outreach
%----------------------------------------------------------------------------------------

\section{Academic service and outreach}
\raggedright\textbf{Outreach}\\
\begin{tabular}{ll}
%---------------------------------------
\textbullet& I have co-organised three \href{http://discoverresearchdublin.com/2014/08/20/night-life/}{Discover Research Night} events in the TCD Zoology Museum\\ 
&aimed at communicating research in evolution and ecology to the general public.\\
&These events have attracted a combined attendance of over 600.\\
\textbullet & I have given several public talks, such as in the Science Gallery Dublin, and\\ 
&I produce videos and images relating to my research \href{http://healyke.github.io/outreach.html}{(see website)}.\\
\textbullet &I am a regular contributor to the\href{http://www.ecoevoblog.com/}{ EcoEvo blog} with one of my posts reaching\\
& the semifinal stages of the \href{http://www.3quarksdaily.com/3quarksdaily/2014/09/3qd-science-prize-semifinalists-2014.html}{3 quirks daily science blog awards}.\\
\textbullet & I have been involved in numerous outreach events including BioBlitz events, PubPhD,\\
&Soapbox Science and I was a finalist in the "I'm a scientist get me out of here" event in 2014.\\
\textbullet & Postgraduate Representative for the Zoology Department 2014-15.\\
\end{tabular}

\raggedright\textbf{Professional society membership}\\
\begin{tabular}{ll}
%---------------------------------------
\textbullet& European Society for Evolutionary Biology (ESEB) and British Ecological Society (BES)\\ % I need more.

\end{tabular}

\raggedright\textbf{Reviewing}\\
\begin{tabular}{ll}
%---------------------------------------
\textbullet&I regularly act as a reviewer for several international academic journals including the Journal of \\
&Animal Ecology, the Journal of Biogeography, Scientific Reports and Proceedings of the Royal\\
&Society B.\\ 

\end{tabular}



\bigskip

\raggedright\textbf{Teaching Experience}\\
\begin{tabular}{ll}
%---------------------------------------
\textit{Teaching and Tutorials}:&I have lectured on both Undergraduate (Evolution) and Masters\\ 
&(Statistics) level courses as well as running several statistics help\\
&workshops and research comprehension courses for Zoology\\
&Senior Sophisters.\\

\end{tabular}


\begin{tabular}{ll}
\textit{Field Course Assistant}:& Field assistant for week long intensive course on ecology for Junior\\ 
&Sophister; teaching field skills in small mammal trapping, insect\\
&and bird identification and ecology field skills.\\
\end{tabular}

\begin{tabular}{ll}
\textit{Project supervision}:&\hspace{5.7mm}Co-supervision of Senior Sophister Zoology student thesis project\\
&\hspace{5.7mm}entitled “Fractal structure of intestinal parasite communities”.\\

\end{tabular}



\bigskip
%-------------------------------------
% References
%--------------------------------------
\section{References}
Contact for References


\bigskip

\end{document}