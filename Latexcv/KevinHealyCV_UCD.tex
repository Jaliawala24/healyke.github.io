% 10/06/2017 
%LaTex version of my CV

\documentclass[10pt,a4paper]{article}
\usepackage[left=0.7in, top=0.6in, right=0.7in, bottom=0.6in]{geometry}
\usepackage{enumerate} % put in numbers or bullet points
\usepackage{setspace}	% line spacing					
\usepackage{eurosym}
%\usepackage{fullpage}
\usepackage{fancyhdr}
\pagestyle{fancy} % page numbers and headers and footers

\usepackage{lipsum}

\newcommand\textbox[1]{%
  \parbox{.333\textwidth}{#1}%
}


\usepackage{amsmath}

\newcommand{\textoperatorname}[1]{%
  \operatorname{\textnormal{#1}}%
}




\fancyhead[RO,RE]{Dr Kevin Healy CV}
\fancyfoot[C]{\thepage~of 4}

\renewcommand{\headrulewidth}{0.1pt}
\renewcommand{\footrulewidth}{0pt}

\renewcommand*{\familydefault}{\sfdefault} %% I changed the font
\usepackage{helvet}

%\usepackage[osf]{mathpazo} % palatino font package
%\usepackage{fontspec} %this must be in XeLaTeX
%\setmainfont{Verdana}%
%\setmainfont{Calibri}%
%\setsansfont{Tahoma}%
%\setmainfont{Helvetica}%
%\setmainfont{Garamond}%
\usepackage{hyperref} % allows the inclusion of hyperlinks




\usepackage{titlesec} % Used to customize the \section command
\titleformat{\section}{\Large\raggedright}{}{0em}{} [\titlerule] % Text formatting of sections
\titlespacing{\section}{0pt}{3pt}{10pt} % Spacing around sections (left, above, below)

\begin{document}

\par{\centering{\Huge \textbf{Dr Kevin {Healy}}}\smallskip\par}

\large\centering{{Marie Curie Research Fellow,\\
Macroecologist at St Andrews University,\\ 
Sir Harold Mitchell Building, St Andrews, UK}}\\
\bigskip

\noindent\textbox{Email: healyke@tcd.ie\hfill}\textbox{\hfil \href{http://scholar.google.com/citations?user=5Kb9u8EAAAAJ}{Google Scholar} \hfil}\textbox{\hfill \href{http://healyke.github.io}{healyke.github.io}}


\bigskip


\section{\textbf{Research Interests}}
\raggedright	
My research focuses on macroecological patterns in the life history and trophic ecology of animals. I use a range of comparative and theoretical approaches across various animal taxonomic groups to test predictions relating to the role of body mass on animal life history strategies and trophic interactions.\\ 
I am also interested in applying  developing new methods within my field and have developed several the R packages.
\bigskip



\section{\textbf{Education and Academic Career}}

\raggedright	
\textbf{2017 - current:} Marie Curie Research Fellow, The University of St Andrews, Scotland.
 \smallskip
\par{\fontsize{10.5}{10} I am currently working on my SCAVENGER project which aims to develop agent based models to understand the role of body size and other drivers on the evolution and ecology of scavenging foraging.
\bigskip}

\raggedright	
\textbf{2015 - 2017:} Research Fellow in Zoology, Trinity College Dublin.
 \smallskip
\par{\fontsize{10.5}{10} I tested questions regarding how body size, ancestral relationships and ecology determine species life history strategies in Prof. Yvonne Buckley's lab.\bigskip}

\raggedright	
\textbf{2011 - 2015:} Ph.D. in Zoology, Trinity College Dublin.\\
Thesis title: “Predator-prey allometry across body size and interaction dimensionality”.\\ \par{\fontsize{10.5}{10} I showed how various ecological and physiological traits define predator-prey interactions. Supervised by Dr. Andrew Jackson and Dr. Andrew Parnell.
\bigskip}

%------------------------------------------

\textbf{2007-2011:} B.A. Mod in Zoology, First class honours (77\%), Trinity College Dublin.\\
Revived a Gold Medal for my final year thesis.
\bigskip

%------------------------------------------
\textbf{2010:} Ureka research position in SoMER program, National College of Ireland Maynooth.\\
\bigskip




%	Awards and Grants
%----------------------------------------------------------------------------------------

\section{\textbf{Awards and Grants}}


\begin{tabular}{ll}
\textbf{2017:} & Marie Sk\l{}odowska-Curie Individual Fellowship (MSCA-IF). Funded by Horizon 2020\\
& (\euro 183,454.80).\\
\textbf{2017:} & Government of Ireland Postdoctoral Fellowship. Funded by the Irish Research\\
& Council (\euro 91,330) *I declined this award to accept the MSCA-IF.\\
\textbf{2014:} & Gordon Research Seminar mentoring program position. Funded by Gordon\\
& Research Conferences and the National Science Foundation. (\euro 1,300)\\
\textbf{2016:} & Winner of the Postdoc category of the Trinity College Dublin, School of Natural\\
& Sciences lightning talks.\\
\textbf{2014:} & Awarded runner up in both the TCD School of Natural Sciences postgraduate\\
& lightning talks and at the joint Zoology and Botany postgraduate symposium.\\
\textbf{2014:} & Awarded runner up in the "I'm a scientist get me out of here" outreach competition\\
\textbf{2014:} & Awarded prize of top contributer in the research discussion group "NERD club".\\
\textbf{2011:} & Awarded Gold medal for “exceptional merit at degree examinations” in final\\
& year of B.A Mod. Zoology by coming first in class and achieving an\\
& overall final year mark of 77\%.\\
\end{tabular}

\bigskip


%-------------------------------------
% Publications
%--------------------------------------
\section{\textbf{Publications}}
\begin{flushleft}


\setlength{\parindent}{0mm}\textbf{Healy, K.,} Guillerme, T., Kelly, S.B.A., Inger, R., Bearhop, S., Jackson, A.L. 2017. SIDER: an R package for predicting trophic discrimination factors of consumers based on their ecology and phylogenetic relatedness. \textit{\textbf{Ecography}}. \href{https://onlinelibrary.wiley.com/doi/abs/10.1111/ecog.03371}{doi:10.1111/ecog.03371 Link to paper}. 6 Google scholar citations.
\bigskip

\setlength{\parindent}{0mm} Adam Kane, \textbf{Healy, K.,}, Guillerme, T., Ruxton, G., and Jackson A.L. 2017. A recipe for scavenging and natural history. \textit{\textbf{Ecography}}. \href{https://onlinelibrary.wiley.com/doi/pdf/10.1111/ecog.02817}{doi:10.1111/ecog.02817 Link to paper}. 4 Google scholar citations.
\bigskip


\setlength{\parindent}{0mm}Kane, A., \textbf{*Healy, K.,} Ruxton, G.D., and Jackson, A.L. 2016. Body size drives importance of scavenging in theropods. \textit{\textbf{The American Naturalist}}. \textbf{6} (187), 706-716. \href{https://www.researchgate.net/profile/Kevin_Healy/publication/301279301_Body_Size_as_a_Driver_of_Scavenging_in_Theropod_Dinosaurs/links/570f8b2a08ae38897ba19c35.pdf.}{DOI: 10.1086/686094 Link to paper}. *As Co-first author I co-conceived the idea and carried out the data collection, analysis and co-wrote of the paper. 4 Google scholar citations.

\bigskip

\setlength{\parindent}{0mm}Donohue, I., Hillebrand, H., Montoya, J.M., Petchey, O.L., Pimm, S.L., Fowler, M.S., \textbf{Healy, K.,} Jackson, A.L., Lurgi, M., McClean, D., O'Connor, N.E., O'Gorman, E.J., Yang, Q. 2016
 A., \textbf{Healy, K.,} Ruxton, G.D., and Jackson, A.L. 2016. Navigating the complexity of ecological stability. \textit{\textbf{Ecology Letters}}. 19 (9), 1172-1185. \href{https://onlinelibrary.wiley.com/doi/abs/10.1111/ele.12648}{doi:10.1111/ele.12648 Link to paper}.
\par{\fontsize{10.5}{10} I contributed by developing methods to calculate a stability score integrating across multiple ecological dimensions. 39 Google scholar citations.}

\bigskip

\textbf{Healy, K}., Guillerme, T., Finlay, S., Kane, A., Kelly, S.B.A., McClean, D., Kelly, D.J., Donohue, I., Jackson, A.L. and Cooper, N., 2014. Ecology and mode-of-life explain lifespan variation in birds and mammals. \textit{\textbf{Proceedings of the Royal Society B}}, \textbf{281}(1784), 20140298. \href{http://rspb.royalsocietypublishing.org/content/281/1784/20140298}{DOI:10.1098/rspb.2014.0298. Link to paper}. This publication received significant media attention both nationally (e.g. the Moncrieff show, Irish Independent) and internationally (Discovery Channel) and has 69 Google scholar citations.

\bigskip

\textbf{Healy, K}., McNally, L, Ruxton, G., Cooper, N. and Jackson, A.L. 2013. Metabolic rate and\\
body size linked with perception of temporal information.  \textit{\textbf{Animal Behaviour}}. \textbf{86}, 685-696. \href{http://dx.doi.org/10.1016/j.anbehav.2013.06.018}{DOI:10.1016/j.anbehav.2013.06.018. Link to paper}. This was extensively covered in the media including coverage from 
\href{https://www.theguardian.com/science/2013/sep/16/time-passes-slowly-flies-study}{The Guardian}, 
\href{http://www.economist.com/news/science-and-technology/21586532-small-creatures-fast-metabolisms-see-world-action-replay-slo-mo}{The Economist},
 an appearance in BBC World News and has the forth highest ever \href{http://www.altmetric.com/details.php?key=517059da36b98ab7d4941284da32e5f7&citation_id=1705703&embedded=true}{alt-metric score} for this journal. This publication has 64 Google scholar citations. 

\bigskip

\setlength{\parindent}{0mm}Donohue, I., Petchey, O.L., Montoya, J.M., Jackson, A.L., McNally, L., Viana, M., \textbf{Healy, K}., Lurgi, M., O’Connor, N.E. and Emmerson, M.C. 2013. On the dimensionality of ecological stability. \textit{\textbf{Ecology Letters}}. \textbf{16}, 421-429. \href{http://onlinelibrary.wiley.com/doi/10.1111/ele.12086/abstract} {DOI:10.1111/ele.12086. Link to paper}. I co-developed the conceptual framework and statistical analysis used to produce the multidimensional ellipsoids and contributed to writing the manuscript.This publication has 108 Google scholar citations.
\bigskip


\textbf{Comment response}\\
\setlength{\parindent}{0mm}\textbf{Healy, K}. 2015.  Eusociality but not fossoriality drives longevity in small mammals. \textit{\textit{Proceedings of the Royal Society B}}, \textbf{282}, 20142917. \href{http://rspb.royalsocietypublishing.org/content/282/1806/20142917} {DOI:10.1098/rspb.2014.2917. Link to paper}.  Single author. I carried out additional analysis in response to a comment on my 2014 paper where I show eusociality but not fossoriality is a driver of longevity in mammals. This publication has 5 Google scholar citations.

\bigskip

\textbf{Statistical packages}\\
\setlength{\parindent}{0mm} Guillerme, T., \textbf{Healy, K.,} 2014. mulTree: a package for running MCMCglmm analysis on multiple trees. ZENODO. DOI: 10.5281/zenodo.12902 {https://github.com/TGuillerme/mulTree Link to package}. This package is hosted on GitHub \href{https://github.com/TGuillerme/mulTree} and has been cited 8 times.

\bigskip

\textbf{Forthcoming Papers}\\

\setlength{\parindent}{0mm}\textbf{Healy, K.,} Ezard, T.H.G., and Jones, O.R., Salguero-Gómez, R., and Buckley, Y.M. Beyond the fast-slow continuum in animal life-histories. In review in \textit{\textbf{Nature}}.
\bigskip


\setlength{\parindent}{0mm}Inger, R., Kelly, D.J., Bearhop, S., Guillerme, T., Hartley, I,R., \textbf{Healy, K.,} Jackson, A.L., Kelly, S.B.A., Mainwaring, M.C., Rowe, L., Sherley, R.B., and Grey, Jonathan. Swallows powered by natural gas? A novel carbon subsidy to terrestrial systems. In review in \textit{\textbf{Nature}}.
\bigskip



\setlength{\parindent}{0mm}\textbf{Healy, K.,} Carbone, C., and Jackson, A.L. Venom evolution in snakes is driven by body size, habitat dimensionality and a diet of eggs. In review in \textit{\textbf{Ecology Letters}}. I show that the evolution of venom toxicity and volume are driven by snake body size, diet and the dimensionality of the habitat.

\bigskip

\section{\textbf{Outreach, and academic service}}
\raggedright\textbf{Outreach}\\
\begin{tabular}{ll}
%---------------------------------------
\textbullet& I gave a TEDxUCD talk on lifespan evolution in animals (\href{https://www.youtube.com/watch?v=-CHtfWEKifY}{link to talk}). I have also given\\
\textbullet& I was presented as part of the documentary \href{https://www.youtube.com/watch?v=-CHtfWEKifY}{documentary}.\\
\textbullet& I co-organised three \href{http://www.ecoevoblog.com/2014/09/22/night-life-friday-26th-sept/}{Discover Research Night} events in the Trinity College Dublin\\
& Zoology Museum with an attendance of 600 over the three events.\\ 
\textbullet& I have been involved in numerous other outreach activities including co-writing a post for\\ 
& the Journal of Animal Ecology \href{https://journalofanimalecology.wordpress.com/2017/09/23/high-society-the-social-network-of-vultures/}{(link)}, presenting at the PubPhD, and partaking in\\
& the 2014 "I'm a scientist get me out of here" event were I was a finalist.\\
\end{tabular}

\raggedright\textbf{Academic service}\\
\begin{tabular}{ll}
%---------------------------------------
\textbullet& I am a committee member for the BES Macroecology special interest group and currently\\
& on the organising team for the BES Macroecology annual meeting in St Andrews 2018.\\
\textbullet& I am co-organising the symposium \href{http://evolutionmontpellier2018.org/symposia}{"Exploring life history evolution across multiple scales"}\\ 
& at the joint Evolution meeting in Montpellier 2018.\\
\textbullet& Committee member for the Irish Ecological Association 2016-2018.\\
\textbullet&Postgraduate representative for the Zoology Department 2014-15.\\
\textbullet&I regularly act as a reviewer for international journals including Ecology Letters,\\
&Current Biology, Proceedings of the Royal Society B, Journal of Biogeography, etc\\
&and for the The German Israeli Foundation for Scientific Research and Development.\\
\end{tabular}

\bigskip

\section{\textbf{Teaching}}
\raggedright\textbf{Undergraduate}\\
\begin{tabular}{ll}
\textbullet& Statistics for biology module for Biodiversity and Conservation masters course in TCD.\\
\textbullet& Contributed to R programming module for third year Biology in St Andrews University.\\
\textbullet& Lecture in Comparative physiology for first year Biology in St Andrews University.\\
\textbullet& Macro-evolution and ecology flipped classroom in final year TCD evolution Module.\\
\textbullet& Introduction to Evolution for 2nd year TCD science students.\\
\textbullet& Research comprehension course for 4th year TCD Zoology.\\
\textbullet& Field assistant for terrestrial ecology field course 3rd year zoology in TCD.\\ 
\end{tabular}
\raggedright\textbf{Post Graduate and workshops}\\
\begin{tabular}{ll}
\textbullet&  I designed and ran a two day Comparative analysis workshop in University College Cork \\
\textbullet&  Module as part of the \href{http://www.demogr.mpg.de/En/education_career/international_advanced_studies_in_demography_3279/past_courses_3280/comparative_approaches_in_ecology_and_evolution_4708/default.htm}{"Comparative Approaches in Ecology and Evolution"} in the Max\\ 
&Plank institute, Rostock, Germany\\
\textbullet&  Teaching Assistant on "Using the COMPADRE Plant Matrix Database\\
& for comparative plant demography" workshop. BES annual meeting, Edinburgh.\\
\end{tabular}
\raggedright\textbf{Student supervision}\\
\begin{tabular}{ll}
\textbullet&  Acted as a thesis supervisor for five final year students between TCD and St Andrews University\\
\textbullet&  I am currently acting as a PhD co-supervisor to a student in TCD.\\
\end{tabular}


\bigskip


%-------------------------------------
% Conferences and workshops
%--------------------------------------


\section{\textbf{Conferences and public speaking}}
\begin{tabular}{ll}
%-------------------------------
\textbf{2016:} & Invited seminar speaker to the Animal and Plant Sciences Department in\\
& The University of Sheffield. "Mapping animal life-history strategies"\\
%-------------------------------
\textbf{2015:} & TEDxUCD invited speaker. \href{https://www.youtube.com/watch?v=-CHtfWEKifY}{"Listening to evolutionary oddities"}.\\
& \href{https://www.youtube.com/watch?v=-CHtfWEKifY}{"Listening to evolutionary oddities"}.\\ 
%-------------------------------
\textbf{2015:} & Invited speaker to the Dublin Science Gallery Cáfe Dark Secrets event.\\ 
& "BIOLUMINESCE: How living organisms produce and emit light"\\ 
%-------------------------------
\textbf{2014:} & Keynote student speaker at the BES Macroecology meeting, Nottingham.\\ 
&"Ecology and mode-of-life explain lifespan variation in birds and mammals".\\
%-------------------------------
\textbf{2014:} & I was an invited speaker to the Irish Longitudinal Study on Aging group (TILDA).\\ 
& "Ecology and mode-of-life explain lifespan variation in birds and mammals".\\
%-------------------------------
\textbf{2014:} & I was an invited speaker to the Dublin Science Gallery Cafe\\ 
& DEAD BEATS event. "Why so venomous?".\\
%-------------------------------
\textbf{2012-2018} & I have presented my work at numerous international conferences over the\\
& years including ESA in Portland 2017, annual BES meetings in 2016 and 2015,\\
& the Gordon Research Seminar "Unifying Ecology Across Scales" in 2014,\\
& the ESEB XIV Congress 2013, EvoDemos in Virgina 2016, IsoEcol in 2012\\
& and the BES Macroecology Sig in 2013, 2014 and 2017\\
%-------------------------------
\end{tabular}

\section{Professional Training}
\begin{tabular}{ll}
\textbf{2016:} & Passport to research futures: XContinuous course aimed at developing skills\\ 
&relating to leading a research team, St Andrews UNiversity.\\
%------------------
\textbf{2016:} & Individual stochastiticy: An introduction to demographic models and analysis,\\ 
&Hal Caswell, University of Virginia.\\
%------------------
\textbf{2015:} & Methods in Ecology and Evolution Workshop on Open Science, Darwin House London.\\
%------------------
\textbf{2014:} & Software Carpentry Workshop covering Unix, Git repositories and creating\\
&R packages, University of Nottingham.\\
%------------------
\textbf{2013:} & Spatial Analysis in R Workshop, Barry Rowlingson, University of Sheffield.\\
%------------------
\textbf{2013:} & Introduction to Morphometrics Workshop, François Gould, Trinity College Dublin.\\
%---------------------
\textbf{2013:} & IUCN Red List of Ecosystems Workshop, Edmund Barrow, Trinity College Dublin.\\
%--------------------
\textbf{2012:} & Introduction to Bayesian analysis using WinBugs, David Lund, University of Cambridge.\\
%--------------------
\textbf{2012:} & Innovation Academy Creative thinking workshop, Trinity College Dublin.\\
%--------------------
\textbf{2012:} & Innovation Academy Film production workshop, Trinity College Dublin.\\
%--------------------
\textbf{2009:} & Mayfly Identification workshop, Mary Kelly Quinn, National Biodiversity Data Centre.\\
\end{tabular}


%-------------------------------------
% Workshops
%--------------------------------------
\section{Working research groups}

\raggedright\textbf{Working research groups}\\
\begin{tabular}{ll}
%---------------------------------------
\textbf{2018:} & Royal Society Disparity workshop: Collaborative meeting developing ideas regarding\\ 
&morphological disparity. Chicheley Hall, UK.\\

\textbf{2018:} & Timespines working group. Linking predation pressures to defensive trait evolution\\ 
\textbf{2018:} & Biogeogrpahy working group. Testing the evolutionary island syndrome effect against isolation effects \\ 



%---------------------------------------
\textbf{2014:} & Tansley Workshop: Collaborative meeting to develop metrics to measure ecosystem\\
&multistabilty, Silwood Park, Imperial College London.\\
%---------------------------------------

%--------------------
&\\
\end{tabular}




\end{flushleft}

%-------------------------------------
% References
%--------------------------------------
%\section{References}


%\begin{tabular}{lcr}
% Referee 1
%\begin{minipage}[t]{2.2in}
%\textbf{Dr\ Andrew L. Jackson}\\
%Zoology Department\\
%School of Natural Sciences\\
%Trinity College Dublin\\
%Dublin 2\\
%Ireland\\
%Tel: + 353 1 896 2728\\
%Email:\href{mailto:a.jackson@tcd.ie}{a.jackson@tcd.ie}
%\end{minipage}

%&

%\begin{minipage}[t]{2.2in}
%\textbf{Dr\ Natalie Cooper}\\
%Life Sciences Department\\
%Natural History Museum\\
%Cromwell Road\\
%London\\
%SW7 5BD UK\\
%Tel: 0207942 5083\\
%Email:\href{mailto:natalie.cooper@nhm.ac.uk}{natalie.cooper@nhm.ac.uk}
%\end{minipage}


%\end{tabular}


%\bigskip

\end{document}